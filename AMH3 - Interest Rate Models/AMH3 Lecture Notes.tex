% Created by Andrew Tulloch

%!TEX TS-program = xelatex
%!TEX encoding = UTF-8 Unicode


\documentclass[10pt, oneside, reqno]{amsbook}
\usepackage{amsthm, amsmath, amssymb}
\usepackage{setspace, graphicx, enumerate}
\onehalfspacing    
\usepackage{fontspec,xltxtra,xunicode}
\defaultfontfeatures{Mapping=tex-text}



% AMS Theorems
\theoremstyle{plain}% default 
\newtheorem{thm}{Theorem}[chapter] 
\newtheorem{lem}[thm]{Lemma} 
\newtheorem{prop}[thm]{Proposition} 
\newtheorem{exer}[thm]{Exercise} 
\newtheorem*{cor}{Corollary} 


\newcommand{\res}[2]{\text{Res}(#1,#2)}
\theoremstyle{definition} 
\newtheorem{defn}[thm]{Definition}
\newtheorem{conj}[thm]{Conjecture}
\newtheorem{exmp}[thm]{Example}

\theoremstyle{remark} 
\newtheorem*{rem}{Remark} 
\newtheorem*{note}{Note} 
\newtheorem{case}{Case} 

\newcommand{\expc}[1]{\mathbb{E}\left[#1\right]}
\newcommand{\var}{\text{Var}}
\newcommand{\cov}[1]{\text{Cov}\left(#1\right)}
\newcommand{\prob}[1]{\mathbb{P}(#1)}
\newcommand{\given}{ \, | \,}
\newcommand{\us}{0 \leq u \leq s}
\newcommand{\ts}[1]{\{ #1 \}}

\renewcommand{\phi}{\varphi}
\newcommand{\sigf}{\mathcal{F}}

\newcommand{\dzz}{\, dz}
\newcommand{\bigo}[1]{\mathcal{O}(#1)}

\newcommand{\al}{\alpha}
\newcommand{\Q}{\mathbb{Q}}
\newcommand{\R}{\mathbb{R}}
\newcommand{\C}{\mathbb{C}}
\newcommand{\Z}{\mathbb{Z}}
\newcommand{\E}{\mathbb{E}}
\newcommand{\N}{\mathbb{N}}

\newcommand{\I}{\mathbb{I}}

\renewcommand{\P}{\mathbb{P}}

\newcommand{\F}{\mathbb{F}}
\newcommand{\Ga}{\mathbb{G}}

\newcommand{\aut}[1]{\text{Aut}{(#1)}}

\newcommand{\gener}[1]{\langle #1 \rangle}
\newcommand{\charr}[1]{\text{char}(#1)}
\newcommand{\nth}{n\textsuperscript{th}}

% \newcommand{\limsup}{\text{limsup}}

\newcommand{\tworow}[2]{\genfrac{}{}{0pt}{}{#1}{#2}}
\newcommand{\xdeg}[2]{[#1 : #2]}
\newcommand{\gal}[2]{\text{Gal}(#1/#2)}
\newcommand{\minpoly}[2]{m_{#1, #2}(x)}

\newcommand{\mapping}[5]{\begin{align*}
	#1 : \quad     #2 &\rightarrow #3 \\
			#4  &\mapsto #5
\end{align*}	
}


\def\cip{\,{\buildrel p \over \rightarrow}\,} 
\def\cid{\,{\buildrel d \over \rightarrow}\,} 
\def\cas{\,{\buildrel a.s. \over \rightarrow}\,} 

\def\clp{\,{\buildrel L^p \over \rightarrow}\,} 

\def\eqd{\,{\buildrel d \over =}\,} 
\def\eqas{\,{\buildrel a.s. \over =}\,}
\newcommand{\sigg}{\mathcal{G}}		
\newcommand{\indic}[1]{\mathbf{1}_{\{ #1 \}} }
\newcommand{\ito}{It\^o\ }		
\newcommand{\itos}{It\^o's\ }		

\newcommand{\doleans}[1]{\mathcal E_t \left(\int_0^\cdot #1 \right)}

\numberwithin{equation}{chapter}

\usepackage{hyperref}
		
\title{AMH3 - Interest Rate Modelling}								% Document Title
\author{Andrew Tulloch}
%\date{}                                           % Activate to display a given date or no date


\begin{document}

\maketitle \tableofcontents \clearpage


\tableofcontents

\chapter{Preliminaries} % (fold)
\label{cha:preliminaries}

% chapter preliminaries (end)
\section{Introduction to Interest Rate Modelling} % (fold)
\label{sec:introduction_to_interest_rate_modelling}

% section introduction_to_interest_rate_modelling (end)
There is a one-to-one correspondence between the class $Q$ of all probability measures equivalent to $\P$ and the class $\Lambda$ of all $\F$-adapted (or $\F$-predictable) process $\lambda_t$ satisfying \[
	\P\left(\int_0^{T^\star} |\lambda_u|^2 \, du < \infty \right) = 1
\]
and \[
	\E_\P\left( \mathcal{E}_{T^\star} \left(\int_0^\cdot \lambda_u \, dW_u\right) \right) = 1
\] 
Thus our correspondence is \[Q \ni \P^\lambda \iff \lambda \in \Lambda.\]
Consequently, \begin{enumerate}[(i)]
	\item $\frac{d\Q}{d\P} = \eta_{T^\star}$
	\item \begin{align*}
		\frac{d\Q}{d\P} \given \sigf_t 	&= \eta^\Q_t \\
										&= \E_\P\left(\eta_{T^\star} \given \sigf_t \right) \\
										&= \doleans{\lambda_u \, dW_u}
	\end{align*}
\end{enumerate}

\begin{thm}[Abstract Bayes formula]
	Let $\Q \sim \P$ with $\frac{d \Q}{d \P} = \eta$.  Suppose that $\sigg \subset \sigf$.  We then have \[
		\E_\Q(X \given \sigg) = \frac{\E_\P(\eta X \given \sigg)}{\E_\P(\eta \given \sigg)}.
	\]  Note that is $\sigg = \{ \emptyset, \Omega \}$ then the formula reduces to \[
		\E_\Q(X) = \E_\P(\eta X).
	\]

	If $Q \sim \P$ with $\frac{d\Q}{d\P} |_{\sigf_t} = \eta_t$, for all $t \in [0, T^\star]$, then \[
		\E_\Q(X \given \sigf_t) = \frac{\E_\P(\eta_{T^\star} X \given \sigf_t)}{\E_\P(\eta_{T^\star} \given \sigf_t)}.
	\]
	
	Hence if $X$ is $\sigf_t$ measurable for some $T \in [0, T^\star]$ then \[
		\E_\Q(X \given \sigf_t) = \frac{\E_\P(\eta_T X \given \sigf_t)}{\eta_t} = \E_\P(\eta_t^{-1} \eta_T X \given \sigf_t)
	\]
\end{thm}

\begin{exmp}
	If $\eta_t = \doleans{\lambda_u \, dW_u}$, then \[
		\E_\Q(X \given \sigf_t) = \E_\P(e^{\int_t^T \lambda_u \, dW_u - \frac{1}{2} \int_0^t |\lambda_u|^2 \, du} X \given \sigf_t)
	\] 
\end{exmp}


\begin{lem}
	\label{lem:martingale_equivalence_measures}
	A $\F$-adapted and $\Q$-integrable process $M$ is a $(\Q, \F)$-martingale if and only if the product $M \eta$ is a $(\P, \F)$-martingale.
\end{lem}

\begin{proof}
	$\E_\Q(M_t \given \sigf_s) = M_s$, $s \leq t$, so \[
		M_s = \E_\Q(M_t \given \sigf_s) = \frac{\E_\P(\eta_t M_t \given \sigf_s)}{\eta_s}
	\]
\end{proof}

\begin{lem}
	If $X$ and $Y$ are two processes of the form \begin{align*}
		dX_t &= \alpha_t \, dt + \beta_t \, dW_t \\
		dY_t &= \tilde \alpha_t \, dt + \tilde \beta_t \, dW_t
	\end{align*} then the product satisfies the \ito product formula \begin{align*}
		d(X_t Y_t) = X_t \, dY_t + Y_t \, dX_t + d \langle X, Y \rangle_t
	\end{align*}	
\end{lem}

If $X$ is of the form $dX_t = \alpha_t \, dt + \beta_t \, dW_t$ and $f$ is of class $C^2(\R)$, then the continuos martingale part of $Y_t = f(X_t)$ is given as \[
	\int_0^t f'(X_u) \beta_u \, dW_u
\]

\begin{prop}
	\label{prop:proposition_1.1}
	% TODO MUST COMPLETE - PROPOSITION 1.1 in course notes.
\end{prop}

\begin{proof}[Proof of Proposition 1.1]
	Let $\P^\lambda$ be equivalent to $\P$, so that \[
		d\eta_t = \eta_t \lambda_t \, dW_t
	\]  and \[
		\frac{d\P^\lambda}{d \P} = \eta_t
	\] on $(\Omega, \sigf_t)$, $t \in [0, T^\star]$.

	Define $B(t, T)$ as follows, for all $t \in [0, T]$, \begin{align*}
		B(t, T) &= B_t \E_{\P^\lambda} \left( \frac{1}{B_T} \given \sigf_t \right) \\
		&= \E_{\P^\lambda} \left(e^{-\int_t^T r_u \, du} \given \sigf_t \right)
	\end{align*}
	
	For i), we simply apply Girsanov's theorem, replacing $dW_t$ by $dW_t = dW^\lambda_t - \lambda_t \, dt$ in the dynamics of $r$ under $\P$.  
	
	For ii), we first recall that $Z(t, T) = \frac{B(t, T)}{B_t}$ is given by \[
		Z(t, T) = \E_{\P^\lambda} \left( \frac{1}{B_T} \given \sigf_t \right)
	\] is a $(\P^\lambda, \F)$-martingale.
	
	Note that $\F^{\lambda} \neq \F$ in general.  From Lemma \ref{lem:martingale_equivalence_measures}, we know that $\eta_t Z(t, T)$ is a $(\P, \F)$-martingale.  Thus applying the predictable representation property, there exists an $\F$-adapted process $\gamma_t$ such that \[
		M_t \equiv \eta_t Z(t, T) = Z(0, T) + \int_0^t \gamma_u \, dW_u
	\] for all $t \in [0, T]$.  Consequently, $dM_t = \gamma_t \, dW_t$ and hence \[
		dZ(t, T) = d(\eta^{-1}_t M_t) = M_t \, d\eta_t^{-1} + \eta_t^{-1}\, dM_t + d \langle \eta^{-1}, M \rangle_t
	\]  where \begin{align*}
		d \eta_t^{-1} = - \eta_t^{-1} \lambda_t \, dW_t^\lambda.
	\end{align*}  We obtain \begin{align*}
		dZ(t, T) &= \eta_t Z(t, T) \left( - \eta_t^{-1} \lambda_t \, dW_t^\lambda \right) + \eta_t^{-1} \gamma_t \left(dW_t^\lambda + \lambda_t \, dt \right) + \left(- \eta_t^{-1} \lambda_t \gamma_t \right) \, dt \\
		&= \eta_t^{-1} \left(\gamma_t - M_t \lambda_t \right) \, dW_t^\lambda
	\end{align*} so that \[
		dZ(t, T) = \tilde b^\lambda(t, T) \, dW_t^\lambda
	\]  Since $B(t, T) = B_t Z(t, T)$, using again the \ito formula we have \begin{align*}
		dB(t, T) 	&= B_t \, dZ(t, T) + Z(t, T) \, dB_t  \\
					&= \frac{B(t, T)}{B_t} r_t B_t \, dt + B_t \tilde b^\lambda(t, T) \, dW_t^\lambda \\
					&= r_t B(t, T) \, dt + B(t, T) \underbrace{\frac{B_t \tilde b^\lambda(t, T)}{B(t, T)}}_{b^\lambda(t, T)} \, dW^\lambda_t.
	\end{align*}   We conclude that for all $T \in [0, T^\star]$, there exists an $\F$-adapted process $b^\lambda(t, T)$, $t \in [0, T]$ called the volatility of the bond
, such that \[
	dB(t, T) = B(t, T)(r_t \, dt + b^\lambda(t, T) \, dW^\lambda_t).  
\]	  In fact, it does not depend on the choice of $\lambda$.  For simplicity, we can write $b(t, T) \equiv b^\lambda(t, T)$.  

The final formula is a special case of the well known result:
\begin{align*}
	dX_t 	&= X_t (\alpha_t \, dt + \beta_t \, dW_t) \\
				\Updownarrow  \\
		X_t	&= X_0 e^{\int_0^t \alpha_u \, du} \doleans {\beta_u \, dW_u }  \\
			&= X_0 e^{\int_0^t \alpha_u \, du} e^{\int_0^t \beta_u \, dW_u - \frac{1}{2} \int_0^t |\beta_u|^2 \, du}
		\end{align*}
		
	This completes our proof of Proposition 1.1, under the assumption that $\frac{1}{B_T}$ is $\P^\lambda$-integrable.
	% We have \[
	% 	dM_t = \eta_t \lambda_t dW_t
	% \] and \[
	% 	dM_t^{-1} = - \eta_t^{-1} \lambda_t dW^\lambda_t.
	% \]  This is true as \[
	% 	\frac{d\P}{d\Q}
	% \]
\end{proof}

There are still several issues given this pricing formula. 
\begin{enumerate}[(i)]
	\item How to compute $b(t, T)$ explicitly in terms of $\mu$ and $\sigma$ under the assumptions that \[
		dr_t = \mu(r_t, t) \, dt + \sigma(r_t, t) \, dW_t
	\] and $\lambda_t = \lambda(r_t, t)$ is the risk premium.  
	\item How can we calibrate our short-term rate model, meaning that \[
		\E_{\P^\lambda}\left( \frac{1}{B_T} \right) = B(0, T) = P(0, T).
	\]
\end{enumerate}  

The issue of pricing bonds is related to solving a backward stochastic differential equation (BSDE).  The general form is \[
	X_t = X_0 + \int_0^t \mu (X_u, u) \, du + \int_0^t \xi_u \, dW_u \tag{$\star$}
\]   where $\mu : \R \times \R^+ \rightarrow \R$ is some function and $\xi$ is some $\F$-adapted process.  We also fix $T > 0$ and postulate that $X_T$ is a \textbf{known } $\sigf_T$-measurable random variable. 

\begin{defn}
	We say that $(X, \xi)$ solves the BSDE with terminal condition with terminal condition $Y$ ($\sigf_T$-measurable) if: 
	\begin{enumerate}[(i)]
		\item $(X, \xi)$ satisfies ($\star$), 
		\item $X_T = Y$.  
	\end{enumerate}
	
	This can be extended to cases where $\mu : \R \times \R^+ \times \Omega \rightarrow \R$ is $\F$-adapted.
\end{defn}


\chapter{Markovian Models of the Short Rate} % (fold)
\label{cha:markovian_models_of_the_short_rate}

	Let $\P^\star$ be a martingale measure in the sense that \[
		B(t, T) = \E_{\P^\star} \left( e^{-\int_t^T r_u \, du} \given \sigf_t \right).
	\]  In particular, \[
		B(0, T) = \E_{\P^\star} \left( e^{-\int_0^T r_u \, du}  \right).  
	\]  We postulate that \begin{equation}
		dr_t = \mu(r_t, t) \, dt + \sigma(r_t, t) \, dW^\star_t,
		\label{eq:generic_short_rate}
	\end{equation} where $W^\star$ is a Brownian motion under $\P^\star$. The filtration $\F$ is any filtration such that $W^\star$ is a BM with respect to $\F$ .  We assume that \eqref{eq:generic_short_rate} has a unique (strong) solution. 
	
	Then it known that $r_t$ has the Markov property with respect to $\F$, meaning that for any bounded continuous function $h : \R \rightarrow \R$, \[
 	\E_{\P^\star} \left(h(r_t) \given \sigf_s \right) = \E_{\P^\star}\left(h(r_t) \given r_s \right)
 \] for all $s \leq t$.   

Hence \[
	\E_{\P^\star} \left(e^{-\int_t^T r_u \, du} \given \sigf_t \right) = v(r_t, t, T) = \tilde v(r_t, t) 
\] suppressing the dependence on $T$.

Goals:
\begin{enumerate}[(i)]
	\item Compute explicitly $v(r_t, t, T)$ for some classical models \begin{enumerate}[(a)]
		\item Merton's model 
		\item Vasicek's model
		\item CIR model (Bessel process)
	\end{enumerate}
	using either the probabilistic approach (martingale measure) or the  analytic approach (PDEs).
	\item Represent the price of the bond as follows \[
		B(t, T) = \exp \left( m(t, T) - n(t, T) r_t \right)
	\]
	
	For a fixed maturity $T$, \[
		m(\cdot, T), n(\cdot, T) : [0, T] \rightarrow \R
	\] can also be computed using the second method by separating variables in the PDE.  Note that $m(T, T), n(T, T) = 0$.
	\item Compute explicitly the volatility $b(t, T)$ of the bond by applying the \ito formula to the function $v(r_t, t, T)$.
	\item Extend the model to the time-inhomogenous case in order to ensure that $B(0, T) = P(0, T)$ for all $T \in [0, T^\star]$.
\end{enumerate}

\section{Merton's model} % (fold)
\label{sub:merton_s_model}
	Assure \[
		r_t = r_0 + at + \sigma W_t^\star 
	\] where $W^\star = W^\lambda$ for some $\lambda$.  Hence
	\begin{equation}
		dr_t = a \, dt + \sigma \, dW^\star_t, \quad r_0 > 0.
	\end{equation} 
	\begin{note}
		The generator of the time homogenous Markov diffusion can be represented as \[
			A_r = a \frac{\partial }{\partial r} + \frac{1}{2} r^2 \frac{\partial^2}{\partial r^2}.  
		\]
	\end{note}

	\begin{prop}
		The price $B(t, T)$ is given by \begin{equation}
			\label{eq:merton_short_rate_price}
			B(t, T) = e^{-r_t(T-t) - \frac{1}{2}a(T-t)^2 + \frac{1}{6} \sigma^2(T-t)^3}.  
		\end{equation}  Hence \[
			dB(t, T) = B(t, T)\left(r_t \, dt - \sigma(T-t) dW^\star_t \right).
		\]  Thus we have the volatility of the bond $b(t, T) = -\sigma(T-t)$.
	\end{prop}
	
	\begin{proof}
		It is enough to calculate $B(0, T)$ and then establish the general formula for $B(t, T)$ using the property that $r_t$ is a time-homogenous Markov process, thus \[
			B(0, T) = v(r_0, T) \Rightarrow B(t, T) =v(r_t, T-t)
		\]
		Computation of $B(0, T)$ is done as follows:\[
			B(0, T) = \E_{\P^\star} \left(e^{-\int_0^T r_u \, du} \right) = \E_{\P^\star} \left( e^{-\xi_T} \right)
		\] where the distribution of $\xi_T$ can be found explicitly.  We argue that \[
			\xi_T \sim N\left(r_0 T + \frac{1}{2} a T^2, \frac{1}{3} \sigma^2 T^3 \right)
		\]  We have \begin{align*}
			\xi_T 	&= \int_0^T r_u \, du \\
					&= \int_0^T \left( r_0 + au + \sigma W^\star_u \right) \, du \\
					&= \int_0^T (r_u + au)  \, du + \sigma \int_0^T W^\star_u \, du
		\end{align*}
		Th rest proceeds quite simply.
		
		We then derive the dynamics of $B(t, T)$.  By the \ito formula, we have that since $B(t, T) = v(r_t, t, T)$, we must have \[
			dB(t, T) = r_t B(t, T) \, dt + b(t, T) B(t, T) \, dW^\star_t.  
		\] Note that the martingale component comes from \[
			\frac{\partial v}{\partial r} dr_t 
		\] and \[
			\frac{\partial}{\partial r}v(r_t, t, T) = -(T- t) v(r_t, t, T)
		\] so that \begin{align*}
			\frac{\partial}{\partial r}v(r_t, t, T) \, dr_t &= -(T-t) v(r_t t, T) (a \, dt + \sigma \, dW^\star_t) \\
			&\sim - \sigma(T- t)B(t, T) \, dW^\star_t
		\end{align*}  We then obtain the equality $B(t, T) = -\sigma(T-t)$.  In particular, $B(t, T) = 0$.  
	\end{proof}
	
	\begin{exer}
		Apply the PDE approach to obtain \eqref{eq:merton_short_rate_price}.
	\end{exer}
	
\section{Vasicek's Model} % (fold)
\label{sub:vasicek_s_model}
Consider the dynamics \begin{equation}
	dr_t = (a - br_t) \, dt + \sigma \, dW^\star_t.
	\label{eq:vasicek_short_rate_dynamics}
\end{equation}

\begin{lem}
	The unique solution to Vasicek's equation is \begin{equation}
		r_t = r_0 e^{-bt} + \frac{a}{b}\left(1-e^{-bt} \right) + \sigma \int_0^t e^{-b(t-u)} \, dW^\star_u.
	\end{equation}
\end{lem}

\begin{prop}
	The bond price in the Vasicek model is given by \begin{align*}
		B(t, T) &= \exp(m(t, T) - n(t, T) r_t) \\
		n(t, T) &= \frac{1}{b}\left(1- e^{-b(T-t)} \right) \\
	\end{align*} and $m(t, T)$ is also known explicitly.
	
	The volatility of the bond satisfies \[
		b(t, T) = -\sigma n(t, T) = -\frac{\sigma}{b}\left(1 -e^{-b(T-t)} \right)
	\] and \[
		dB(t, T) = B(t, T) \left(r_t \, dt - \sigma n(t, T) \, dW^\star_t \right).
	\]
\end{prop}

\begin{thm}[Stochastic Fubini's theorem]
	In the computation above, we obtain the following double integral \[
		\int_0^T \int_0^t e^{-b(t-u)} \, dW^\star_u \, dt = \frac{1}{b} \int_0^T \left(1-e^{-b(T-u)} \right) \, dW^\star_u.
	\]   To obtain this result, we must use the stochastic Fubini theorem \[
		\int_0^T \int_0^t f(t, u) \, dW^\star_u \, dt = \int_0^T \int_u^T f(t, u) \, dt \, dW^\star_u
	\] where $f$ is a continuous function.

\end{thm}

\subsection{PDE Approach to Vasicek's model} % (fold)
\label{sub:pde_approach_to_vasicek_s_model}
We can either use some known results or provide some simple arguments.  

We start by postulating that $B(t, T) = v(r_t, t, T)$ where $v \in C^{2,1}(\R \times [0, T\star], \R)$.  On the other hand, we may apply the \ito formula and obtain \begin{align*}
	dv(r_t, t, T) = \left( \frac{\partial r}{\partial t} + \mu(r_t, t) \frac{\partial v}{\partial r} + \frac{1}{2} \sigma^2(r_t, t) \frac{\partial^2 v}{\partial r^2} \right) \, dt + \sigma(r_t, t) \frac{\partial v}{\partial r} dW^\star_t.
\end{align*} 

 On the other hand, from Proposition \ref{prop:proposition_1.1} we have \[
 	dB(t, T) = dv(r_t, t, T) = r_t v(r_t, t, T) \, dt + b(t, T)v(r_t, t, T) \, dW^\star_t.
 \]  This means that \begin{align*}
 	\underbrace{\left( \frac{\partial r}{\partial t} + \mu \frac{\partial v}{\partial r} + \frac{1}{2} \sigma^2 \frac{\partial^2 v}{\partial r^2} - r_t v \right) \, dt}_{A_t} &= \underbrace{\left( b(t, T) v - \sigma \frac{\partial v}{\partial r} \right) \, dW^\star_t}_{M_t}.
 \end{align*}

\begin{lem}
	If $(M_t)_{t \in [0, T^\star]}$ is a continuous local martingale and a process of finite variation then $M_t = M_0$ for $t \in [0, T^\star]$.
\end{lem}

Since $r_t$ is a Gaussian process, we note that the unknown function should necessarily satisfy the following pricing PDE for $v = v(r_t, t, T)$, \[
	\begin{cases}
		 \frac{\partial r}{\partial t} + \mu \frac{\partial v}{\partial r} + \frac{1}{2} \sigma^2 \frac{\partial^2 v}{\partial r^2} - r_t v = 0 \\
		v(r_t, T, T) = h(r_t).
	\end{cases}
\] For the bond maturing at $T$, we set $h(r) = 1$.

To solve this PDE in the Vasicek case, we postulate that \[
	v(r_t, t, T) = e^{m(t, T) - n(t, T) r_t}
\] and derive a system of two ODEs satisfied by the function $m$ and $n$. 
% subsection pde_approach_to_vasicek_s_model (end)
% subsection vasicek_s_model (end)
	% 
	% Recall that \begin{align*}
	% 	B(t, T) = \E_{\P^\star} \left( \frac{B_t}{B_T} \given \sigf_t \right) = f(r_t, t, T) \\
	% 	dB_t = r_t B_t \, dt, \quad B_0 = 1.
	% \end{align*}  
	% 
	% Computations of $f$ will be mainly based on simple properties of the Gaussian distribution. 
	% 
	% \begin{lem}
	% 	Assume that $\xi \sim N(\mu, \sigma^2)$.  Then $\zeta = e^{\xi}$ has the expected value \[
	% 		\E(\zeta) = e^{\mu + \frac{1}{2}\sigma^2}, \quad \var(\zeta) = e^{2 \mu + \sigma^2}(e^{\sigma^2} - 1).
	% 	\]
	% \end{lem}
	% 
	% \begin{proof}
	% 	We have \begin{align*}
	% 		\E(\zeta) = \E(e^{\xi}) = \E(e^{\mu + \sigma Z}) = e^\mu \E(e^{\sigma Z}) = e^{\mu} \E(e^{\sigma Z - \frac{1}{2} \sigma^2 + \frac{1}{2} \sigma^2}) = e^{\mu} e^{\frac{1}{2} \sigma^2} \underbrace{\E(e^{\sigma Z - \frac{1}{2} \sigma^2 t})}_{=1} = e^{\mu + \frac{1}{2}\sigma^2}
	% 	\end{align*}
	% \end{proof}

% subsection merton_s_model (end)

% lecture_4 (end)

\section{Valuation of Bond Options} % (fold)
\label{sub:valuation_of_bond_optiosn}

Consider a European call option on a $U$-maturity zero-coupon bond with expiry $T$ and strike $K$ where $t \leq T < U$ and $K > 0$.  The payoff at time $T$ equals \[
	C_T = \left(B(T, U) - K \right)^+ = \left(B(T, U) - KB(T, T) \right)^+
\]  We postulate that \begin{align*}
	C_t 	&= B_t \E_{\P^\star} \left(B_T^{-1} C_T \given \sigf_t \right) \\
			&= \E_{\P^\star} \left( e^{-\int_t^T r_u \, du} \left(v\left(r_T, T, U \right) - K\right)^+ \right)	
\end{align*}

The idea is to change the martingale measure $\P^\star$ to another probability measure $\Q$ such that \begin{align*}
	C_t &= B(t, T) \E_{\Q} \left(C_T \given \sigf_t \right) \\
	&= B(t, T) \E_{\Q} \left( \left(F_t \xi - K \right)^+ \given \sigf_t \right)
\end{align*} where $F_t = \frac{B(t, U)}{B(t, T)}$.  The measure $\Q$ is equivalent to $\P$ on $(\Omega, \sigf_T)$ and it is chosen in such a way such that $(F_t)_{t \in [0, T]}$ is a $\Q$-martingale.


Alternatively, consider a claim $X = C_T$ maturing at time $T$.  Then \begin{align*}
	C_t &= B_t \E_{\P^\star} \left( \frac{C_T}{B_T} \given \sigf_t \right) \\
	\Phi_t(X) &= B_t \E_{\P^\star} \left( \frac{X}{B_T} \given \sigf_t \right)
\end{align*}
\begin{exmp}
	In the context of equity options this approach yields the following representation of the price of a call option: \[
		C_t = S_t \hat P \left(S_T > K \given \sigf_t \right) - K B(t, T) \P^\star \left( S_T > K \given \sigf_t \right)
	\] 
	where \begin{align*}
		\frac{B_t}{S_t} &\text{ is a $\hat P$-martingale} \\
		\frac{S_t}{B_t} &\text{ is a $P^\star$-martingale}
	\end{align*}  If $B_t = e^{rt}$ (deterministic) then $\hat P =  P^\star$.
\end{exmp}

\section{The CIR Model} % (fold)
\label{sub:the_cir}
We postulate that \[
	dr_t = (a- b r_t) \, dt + \sigma \sqrt{r_t} \, dW^\star_t,
\]  where $a, b \sigma$ are positive constants.  Using Yamada-Watanabe theorem, we obtain uniqueness and existence of solutions.  A suitable comparison theorem tells us that if $r_0 > 0$ then $r_t \geq 0$ for $ t \in [ 0, T]$.  It is known that the solution $r$ to the CIR equation is related to the Bessel process.  It is known that \begin{enumerate}[(i)]
	\item $B(t, T) = e^{m(t, T) - n(t, T) r_t}$ where $m$ and $n$ can be computed explicitly using the PDE approach.  
	\item The price of a call option can be computed explicitly using the probabilistic approach.
\end{enumerate} 

One can prove that \[
	C_t = B(t, U) \Phi_1(B(t, U), B(t, T), t, T, U) - K B(t, T) \Phi_2(B(t, U), B(t, T), t, T, U)
\] where $\Phi_1, \Phi_2$ are given explicitly in terms of the distribution of a Bessel process.

\section{Calibration} % (fold)
\label{sub:calibration}

We denote by $\hat B(0, T)$ the market price of a zero coupon bond with maturity $T$.  We assume that \[
	\hat B(0, T) = e^{- \int_0^T \hat f(0, u) \, du}
\] where the instantaneous forward rate  is a differentiable function such that \[
	\hat f_T(0, t) 
\] exists for $t \in 0, T$.  In general, we can fit to market data a model of the form \[
	dr_t = (a(t) - br_t) \, dt + \sigma r_t^\beta \, W^\star_t
\] for $\beta \in [0, 1]$.  

\begin{prop}
	Let $\beta = 0$. Then the model fits the market data if and only if $a(t) = \hat f_T(0, t) + h'(t) + b(\hat f(0, t) + h(t))$ where \[
		h(t) = \frac{\sigma^2\left(1-e^{-bt}\right)^2}{2b^2}.
	\]
\end{prop}
It is essential here to assume that the function $\hat f(0, T)$ is differentiable with respect to $T$.  If we wish to produce a model such that $f(0, T) = \hat f(0, T)$.

% subsection calibration (end)
% subsection the_cir (end)
% subsection valuation_of_bond_optiosn (end)

\chapter{The HJM Approach to Modelling Bond Prices} % (fold)
\label{cha:the_hjm_approach_to_modelling_bond_prices}

\section{Introduction} % (fold)
\label{sec:introduction_to_hjm}

% section introduction (end)
Take as inputs the following objects \begin{enumerate}[(i)]
	\item $(\Omega, \F, \P)$, $W$, a $d$-dimensional Brownian motion.
	\item The dynamics of a family of processes \[
		\{ f(t, T), t \in [0, T] \}, T \in [0, T^\star]
	\] where $f(\cdot, T)$ is an $\F$-adapted process such that \[
		df(t, T) = \alpha(t, T) \, dt + \sigma(t, T) \cdot dW_t
	\] with some initial condition $f(0, \cdot) : [0, T^\star] \rightarrow \R$.  
\end{enumerate} As an output, we obtain the family of bond prices \[
	\{ B(t, T), t \in [0, T] \}, T \in [0, T^\star]
\] given by \[
	B(t, T) = \exp \left(-\int_t^T f(t, u) \, du \right)
\]  We must first derive the dynamics of $B(\cdot, T)$ under $\P$ for any maturity $T$ in the following form \[
	dB(t, T) = B(t, T) \left( a(t, T) \, dt + b(t, T) dW^\star_t \right)
\] where $a$ and $b$ are given in terms of $\alpha$ and $\beta$.  

Next, we will find out under which assumptions on $\alpha$ and $\beta$ the HJM model admits a spot martingale measure $\P^\star$ or equivalently, a forward martingale measure $\P_{T^\star}$.  

By definition, $\P^\star$ is any probability measure on $(\Omega, \sigf_{T^\star})$ such that $\P^\star \sim \P$ and the processes \[
	Z_t = \frac{B(t, T)}{B_t} = \frac{B(t, T)}{\exp\left(\int_0^t f(u, u) \, du \right)}
\] are $\P^\star$-(local) martingales.  Similarly, $\P_{T^\star} \sim \P$ and the processes \[
	F(t, T, T^\star) = \frac{B(t, T)}{B(t, T^\star)} 
\] are $\P_{T^\star}$-(local) martingales.

\begin{note}
	Let $F(t, T, U) = \frac{B(t, T)}{B(t, U)}$.
	
	\begin{enumerate}[(i)]
		\item If $U \leq T$, then $F(t, T, U)$ is the forward price of a $T$-maturity bond for the settlement date at time $U$.
		\item If $U \geq T$ then $F(t, T, U)$ represents the forward rate in the FRA initiated at time $t$ for the future time interval $[T,U]$. 
	\end{enumerate}  
\end{note}
	
\begin{defn}[HJM approach]
	Assume that \[
		df(t, T) = \alpha(t, T) \, dt + \sigma(t, T) \cdot dW_t
	\] with $W$ a $d$-dimensional Brownian motion and \[
		\sigma(t, T) \cdot dW_t = \sum_{i=1}^d \sigma^{i}(t, T) dW_t^{i}. 
	\]  All processes are specified under $\P$.  
	
	We define $B(t, T) = e^{-\int_t^T f(t, u) \, du}$.  
\end{defn}


\begin{lem}
	Let $\alpha^\star(t, T) = \int_t^T \alpha(t, u) \, du$, and $\sigma^\star(t, T) = \int_t^T \sigma(t, u) \, du$.  These are $\F$-adapted processes.
	
	Then we claim that \[
		dB(t, T) = B(t, T) \left( a(t, T) \, dt + b(t, T) \cdot dW_t \right)
	\] where \begin{align*}
		a(t, T) &= f(t, t) - \alpha^\star(t, T) + \frac{1}{2} \left(\sigma\star(t, T)\right)^2 \\
		b(t, T) &= - \sigma^\star(t, T).
	\end{align*}
\end{lem}

Let $Z(t, T) = \frac{B(t, T)}{B_t}$, with $B_t = e^{\int_0^t f(u, u) \, du}$, so that \[
	dZ(t, T) = Z(t, T) \left( \left(\frac{1}{2} \left( \sigma(t, T) \right)^2 - \alpha^\star(t, T) \right) \, dt - \sigma^\star(t, T) \cdot dW_t \right) 
\]  

Under which assumptions on $\alpha$ and $\sigma$ does there exists a probability measure $\Q \sim \P$ on $(\Omega, \sigf_{T^\star})$ such that $Z(t, T), t \in [0, T]$ is a $\Q$-martingale for every $T \in [0, T^\star]$.

We can also form process \[
	F_B(t, T, T^\star) = F(t, T, T^\star) = \frac{B(t, T)}{B(t, T^\star)}.
\]

\section{Trading Strategies} % (fold)
\label{sec:trading_strategies}
We first choose $\tau = \{ T_1 < T_2 < \dots < T_k \leq T^\star \}$ and take some $\F$-adapted process $\phi = (\phi^1, \dots, \phi^k)$.  $\tau$ represents the maturities of traded bonds.  $\phi^i$ represents the number of shares of $\tau_i$-maturity bonds.  

Then the wealth process of $(\phi, \tau)$ equals \[
	V_t(\phi) = \sum_{i=1}^k \phi^i_t B(t, T_i).
\] 

\begin{defn}[Self-financing]
	We say that $\phi$ is self financing if \[
		dV_t(\phi) = \sum_{i=1}^k \phi_t^i \, dB(t, T_i).
	\]
\end{defn}

\begin{lem}{\ }
	\begin{enumerate}[(i)]
		\item 
	Let $V_t^\star(\phi) = \frac{V_t(\phi)}{B_t}$.  Then $\phi$ is self-financing if and only if \[
		dV^\star_t(\phi) = \sum_{i=1}^k \phi_t^i d Z(t, T_i).
 	\] 
		\item 
		Let $F_v(t, T) = \frac{V_t(\phi)}{B(t, T)}$ for some $0 < T \leq T^\star$. Then $\phi$ is self-financing if and only if \[
			dF_v(t, T) = \sum_{i=1}^k \phi^i_t d \left( \frac{B(t, T_i)}{B(t, T)} \right) = \sum_{i=1}^k  \phi^i_t dF(t, T_i, T)
		\] where we assume $T \geq T_k$.
	\end{enumerate}

\end{lem}
% section trading_strategies (end)

\section{Martingale Measures} % (fold)
\label{sec:martingale_measures}
We will first address the issue of existence of the so-called \emph{forward martingale measure}, that is, a martingale measure for processes $\frac{V_t(\phi)}{B(t, T^\star)}$ or equivalently, a martingale measure for processes \[
	F_B(t, T, T^\star) = \frac{B(t, T)}{B(t, T^\star)}, t \in [0,T], T \in [0, T^\star].
\]  

\begin{lem}
	For any $T \in [0, T^\star]$, \begin{align*}
		dF_B(t, T, T^\star) = F_B(t, T, T^\star) \left( \tilde a(t, T) \, dt +  (b(t, T) - b(t, T^\star) ) \, dW_t \right) 
	\end{align*} where \[
		\tilde a(t, T) = a(t, T) - a(t, T^\star) - b(t, T^\star) \left( b(t, T) - b(t, T^\star) \right)
	\]
	
	We denote by $\hat \P = \P^\star$ the martingale equivalent to $\P$ on $(\Omega, \sigf_{T^\star})$ by \[
		\frac{d\hat P}{d \P} = \doleans{h_u \, dW^\star_u}
	\]  If $h$ is such that \[
		\E\left(\mathcal E_{T^\star} \left( \int_0^\cdot h_u \, dW_u \right) \right) = 1
	\] the $\hat \P$ is well defined and we can compute the dynamics of $F_B(t, T, T^\star)$ under $\hat P$ with respect to $\hat W$, where \[
		\hat W = W_t - \int_0^t h_u \, du, t \in [0, T^\star]
	\]
	
	Assume that \begin{equation}
		a(t, T) - a(t, T^\star) = (b(t, T^\star) - h_t) \cdot (b(t, T) - b(t, T^\star))
		\label{eq:3.10_in_lecture_notes}
	\end{equation}  Condition \eqref{eq:3.10_in_lecture_notes}  in the lecture notes ensures that there is no drift term in the dynamics of $F_B(t, T, T^\star)$ under $\hat P$ for all maturities $T$.  After some computations, \eqref{eq:3.10_in_lecture_notes} can be represented as follows
	\[
	\alpha(t, T) + \sigma(t, T) \left(h_t + \int_T^{T^\star} \sigma(t, u) \, du \right) = 0.
	\]
\end{lem}

Later on we will denote by $\P_T$ the forward measure for the date $T$.  Thus $\hat P = \P_{T^\star}$.

% section martingale_measures (end)

\subsection{Spot Martingale Measure} % (fold)
\label{sec:spot_martingale_measure}
We know that \[
	dZ(t, T) = - Z(t, T) \left( \left(\alpha^\star(t, T) - \frac{1}{2} \left| \sigma^\star(t, T) \right|^2 \right) \, dt + \sigma^\star(t, T) \, dW_t \right)
\]

Now, the conditions for the drift term in $dZ(t, T)$ disappearing reads \begin{align*}
	\alpha^\star(t, T) &= \frac{1}{2} |\sigma^\star(t, T) |^2 - \sigma^\star(t, T) \lambda_t \\
	\Updownarrow  \\
	\alpha(t, T) &= \sigma(t, T) \left(\sigma^\star(t, T) - \lambda_t \right)
\end{align*}

The last formula can be seen as a tool for simple derivations of processes of interest interest under the measure $\P^\star$ (setting $\lambda = 0$).  We denote \[
	W^\star_t - W_t - \int_0^t \lambda_u \, du
\] 
% subsection spot_martingale_measure (end)

\subsection{Forward Measure} % (fold)
\label{sub:forward_measure}
We are going to examine the relationship between $\P^\star$ and $\P_T$ in a general term structure model.

\begin{note}
	Define the following.
	\begin{align*}
		dB(t, T) &= B(t, T) \left(r_t \, dt + b(t, T) \, dW^\star_t \right) \\
		d \zeta^i_t &= \zeta^i_t \left(r_t \, dt + \sigma^i_t \, dW^\star_t \right)
	\end{align*}
\end{note}

By definition, \[
	\pi_t(X) = B_t \E_{\P^\star} \left( \frac{X}{B_T} \given \sigf_t \right)
\]  Can we fine $\Q$ such that $\Q \sim \P^\star$ and \[
	B_t \E_{\P^\star} \left( \frac{X}{B_T} \given \sigf_t \right) = B(t, T) \E_\Q \left( X \given \sigf_t \right)
\] for any claim $X \in \sigf_T$ where $B(t, T) = B_t\E_{\P^\star} \left( \frac{1}{B_T} \given \sigf_t \right)$.  Formally, \[
	\E_{\Q} \left( X \given \sigf_t \right) = \frac{\E_{\P^\star} \left( \frac{X}{B_T} \given \sigf_t \right)}{\E_{\P^\star} \left( \frac{1}{B_T} \given \sigf_t \right)}
\]

We are guessing that $\Q \sim \P^\star$ with density on $(\Omega, \sigf_t)$ \[
	\frac{d\Q}{d \P} = \frac{1}{B(0, T) B_T}, \P^\star-a.s.
\]  $\E_{\P^\star} \left(\frac{1}{B_T} \right) = B(0, T)$.

\begin{defn}
	Suppose that $\P^\star$ is a spot martingale measure for our model.  Then for any maturity $T \in [0, T^\star]$, we define the forward martingale measure for the date $T$ by setting on $(\Omega, \sigf_{T^\star})$ \[
		\frac{d\P_T}{d\P^\star} = \frac{1}{B(0, T) B_T}, \P^\star-a.s.
	\]
\end{defn}

\begin{prop}{\ }
	\begin{enumerate}[(i)]
		\item \begin{align*}
			\frac{d\P_T}{d\P^\star} \given \sigf_t &= \E_{\P^\star} \left( \frac{d\P_T}{d\P^\star} \given \sigf_t \right) \\
				&= \E_{\P^\star} \left( \frac{B_0 B(T, T)}{B(0, T) B_T} \given \sigf_t \right) \\
				&= \frac{B_0}{B(0, T)} \E_{\P^\star} \left( \frac{B(T, T)}{B_T} \given \sigf_t \right) \\
				&= \frac{B_0}{B(0, T)} \frac{B(t, T)}{B_t}, \P^\star-a.s.
		\end{align*}
	\end{enumerate} 
\end{prop}

Recall that $\frac{\pi_t(X)}{B_t}$ is a $\P^\star$-martingale. Similarly, $\frac{\pi_t(X)}{B(t, T)}$ is a $\P_T$-martingale.  If $\eta_t = \frac{d \P_T}{d\P^\star} \given \sigf_t$ then $M$ is a $\P_T$-martingale if and only if $M \eta$ is a $\P^\star$-martingale.  

\begin{exer}
	If we know that under $\P$ processes $\frac{X_t}{Z_t}$ are martingales where $Z$ is a fixed, positive process and under $\Q$ process $\frac{X_t}{Y_t}$ are martingales for a fixed positive process $Y$ then we can find a density of $\Q$ with respect to $\P$ in terms of $Z$ and $Y$. 
\end{exer}


We consider an arbitrage free model of bond prices and stock prices in which the spot martingale measure $\P^\star$ exists, such that $\frac{B(t, T)}{B_t}$ and $\frac{S^i_t}{B_t}$ are $\P^\star$-martingales.

We do not postulate that our model is complete.

Assume that $X$ is an attainable claim in this model.  We know that the arbitrage price $\pi_t(X)$ is unique and it can be computed using the risk-neutral valuation formula \[
	\pi_t(X) = B_t \E_{\P^\star} \left( \frac{X}{B_T} \given \sigf_t \right).
\]

\begin{rem}
	How do we find the forward price of $X$ at the time $t$ in the forward contract with settlement date $T$.
\end{rem}

\begin{defn}[Forward contract]
	The forward contract written at time $t$ on a time $T$ contingent claim is represented by the time $T$ contingent claim \[
		G_T = X - F_X(t, T)
	\] such that \begin{enumerate}[(i)]
		\item $F_X(t, T)$ is an $\sigf_t$-measurable random variable, 
		\item the arbitrage price at time $t$ on a contingent clam $G_T$ equals zero, that is, $\pi_t(G_T) = 0$.
	\end{enumerate}
\end{defn}

To compute $F_X(t, T)$, we will use the risk-neutral formula

\begin{align*}
	\pi_t(G_T) &= B_t \E_{\P^\star} \left( \frac{G_T}{B_T} \given \sigf_t \right) \\
		&= B_t \E_{\P^\star} \left( \frac{X}{B_T} \given \sigf_t \right) - F_X(t, T) B_t \E_{\P^\star} \left(\frac{1}{B_T} \given \sigf_t \right) \\
		&= \pi_t(X) - F_X(t, T) B(t, T) \\
		&= 0
\end{align*} and so \[
	F_X(t, T) = \frac{\pi_t(X)}{B(t, T)}.
\] 

Define \begin{align*}
	F_Z(t, T) &= \frac{Z_t}{B(t, T)} \qquad \text{$Z_t = S_t$ or $B(t, T)$} \\
	F_S(t, T) &= \frac{S_t}{B(t, T)} \qquad \text{forward price of stock $S$}\\
	F_B(t, U, T) &= \frac{B(t, U)}{B(t, T)} \qquad \text{forward price of $U$-maturity bond}.
\end{align*}

\begin{defn}[Forward measure]
	
	We assume that $\P^\star$ is given.  The corresponding forward measure for the date $T, T \in [0, T^\star]$ is defined by \[
		\frac{d\P_T}{d\P^\star} = \frac{1}{B(0,T) B_T}, \quad \P^\star-a.s.
	\] so that \[
		\frac{d\P_T}{d\P^\star} \, |_{\sigf_t} = \frac{B_0}{B(0, T)} \frac{B(t, T)}{B_t}
	\] for every $t \in [0, T]$.
\end{defn}

\begin{lem}
	Assume that $W_t^\star$ is a Brownian motion under $\P^\star$ and \[
		dB(t, T) = B(t, T) \left( r_t \, dt + b(t, T) \, dW^\star_t \right)
	\]  Then $\eta_t \equiv \frac{d\P_T}{d\P^\star} \, |_{\sigf_t}$ equals \[
		\eta_t = \exp \left( \int_0^t b*u, T) \, dW^\star_u - \frac{1}{2} \int_0^t \left| b(u, T) \right|^2 \, du \right).
	\]  That is, \[
		\eta_t = \doleans{b(u, T) \, dW^\star_u}. 
		\tag{$\star$}
	\]  
	
	It then follows that \[
		d \eta_t = \eta_t b(t, T) \, dW^\star_t, \quad \eta_0 = 1.
	\] and \[
		W_t^T = W^\star_t - \int_0^t b(u, T) \, du
	\] is a Brownian motion under $\P_T$.
\end{lem}

\begin{proof}
	Equation $(\star)$ follows from \[
		\frac{d\P_T}{d\P^\star} \, |_{\sigf_t} = \frac{B_0}{B(0, T)} \frac{B(t, T)}{B_t}
	\]  
	
	The corollaries follow from differentiation and Girsanov's theorem, respectively.
\end{proof}

\begin{exer}
	Let $T \leq U$.  Find the dynamics of the forward price $F_B(t, U , T)$ under $\P_T$.  Apply the \ito formula under $\P^\star$, use Girsanov's theorem to express the dynamics of $F_B(t, U, T)$ in terms of $b(t, T)$, $b(t, U)$ and $W^T$.  Compute the volatility $\gamma(t, U ,T)$ of $F_B(t, U, T)$.  Apply the above the the HJM model $(\alpha(t, T), \sigma(t, T), W)$.
\end{exer}

\subsection{Applications of forward measures} % (fold)
\label{sub:applications_of_forward_measures}
\begin{enumerate}[(i)]
	\item Valuation of contingent claims.
	\item Construction of models for market rates.
\end{enumerate}  Application (i) is based on the following equality \[
	B_t \E_{\P^\star} \left( \frac{X}{B_T} \given \sigf_t \right) = B(t, T) \E_{\P_T} \left( X \given \sigf_t \right).
\]

\begin{lem}
	If $X$ is an attainable claim and settles at time $T$, then \[
		\pi_t(X) = B(t, T) \E_{\P_T} \left( X \given \sigf_t \right)
	\]
\end{lem}

\subsubsection{Valuation of claims with maturity $U \neq T$.}
Assume that $U \leq T$. Then the payoff $X$ at $U$ is equivalent to the payoff $Y = \frac{X}{B(U, T)}$ at time $T$.  Equivalence is understood in the sense that \[
	\text{$X$ at $U$} \sim \text{$Y$ at $T$} \iff \pi_t(X) = \pi_t(Y), t \in [0, U].
\]  So \[
	\pi_t(X) = B(t, U) \E_{\P_U} \left( X \given \sigf_t \right) = \pi_t(Y) = B(t, T) \E_{\P_T} \left( \frac{X}{B(U, T)} \given \sigf_t \right).
\]

To establish this equality, observe that for $t \in [0, U]$, \begin{align*}
	\frac{d \P_U}{d \P_T} \, |_{\sigf_t} = \frac{ \frac{d\P_U}{d\P^\star} |_{\sigf_t}}{\frac{d\P_T}{d\P^\star} |_{\sigf_t}} = \frac{ \frac{B_0}{B(0, U)} \frac{B(t, U)}{B_t}}{\frac{B_0}{B(0, T)}\frac{B(t, T)}{B_t}} = \frac{B(0, T)}{B(0, U)}\frac{B(t, U)}{B(t, T)}.
\end{align*}  We then need only apply the Bayes formula and apply the previous result.

Assume now that $U \geq T$. We postulate that $X$ is $\sigf_T$-measurable. Then the claim $Y = B(T, U) X$ is equivalent to $X$, in the sense that $\pi_t(X) = \pi_t(Y)$.

\begin{enumerate}[(i)]
	\item $U \leq T$.  Then $\pi_t(X) = B(t, T) = \E_{\P_T} \left( \frac{X}{B(U, T)} \given \sigf_t \right)$.  
	\item $U \geq T$ and $X \in \sigf_T$.  Then $\pi_t(X) = B(t, T) \E_{\P_T} \left( B(T, U) X \given \sigf_t \right)$.
\end{enumerate}


% subsection applications_of_forward_measures (end)

\section{The Gaussian HJM Model} % (fold)
\label{sec:examples_of_hjm_models}
Under $\P^\star$, \begin{equation}
	\label{eq:hjm_1}
	dB(t, T) = B(t, T) \left(r_t \, dt - \sigma^\star(t, T) \, dW^\star_t \right)
\end{equation} where \begin{equation}
	\label{eq:hjm_2}
	-\sigma^\star(t, T) = \int_t^T \sigma(t, u) \, du = b(t, T).
\end{equation}  Moreover, \begin{equation}
	\label{eq:hjm_3}
	df(t, T) = \sigma(t, T) \sigma^\star(t, T) \, dt  + \sigma(t, T) \, dW^\star_t
\end{equation} and \begin{equation}
	\label{eq:hjm_4}
	r_t = f(0, t) + \int_0^t \sigma(u, t) \sigma^\star(u, t) \, du + \int_0^t \sigma(u, t) \, dW^\star_u
\end{equation} 

\begin{rem}
	From \eqref{eq:hjm_1} and \eqref{eq:hjm_3}, we see that for any fixed $T$, processes $B(t, T)$ and $f(t, T)$ are continuous semimartingales.  In \eqref{eq:hjm_4}, we integrate a different process for each $t$.  Also, as an additional input we take some function $f(0, t)$.
	
	Can we then compute $dr_t$?  The answer to this question is positive in some special cases.
\end{rem}

We now always postulate that $\sigma(t, T)$ is deterministic.  Then we say that we deal with the \emph{Gaussian HJM model} since $r_t$ has a normal distribution for any $t \in [0, T^\star]$.  

Several examples of the Gaussian HJM model include: 
\begin{enumerate}[(i)]
	\item The Ho-Lee model.  We take $d = 1$ and $\sigma(t, T) = \sigma$.  Since $b(t, T) = - \sigma(T-t)$, it can also be seen as a counterpart to Merton's model.
	\item The bond price satisfies under $\P^\star$, \begin{align*}
		dB(t, T) = B(t, T) \left( r_t \, dt - \sigma(T- t) \, dW^\star_t \right).
	\end{align*}  The short term rate equals\[
		r_t = f(0, t) + \frac{1}{2} \sigma^2 t^2 + \sigma W^\star_t,
	\] so that \[
		dr_t = \underbrace{\left( f_T(0, t) + \sigma^2 t \right)}_{a(t)} \, dt + \sigma dW^\star_t .
	\] where the function $a : [0, T^\star] \rightarrow \R$ can also be derived if we start from the extended merton model $dr_t = a(t) \, dt + \sigma dW^\star_t$ and we fit this model to the yield curve $\E_{\P^\star} \left( e^{-\int_0^T r_t \, dt} \right) = e^{-\int_0^T f(0, t) \, dt}$.  We also need to show that $r_0 = f(0, 0)$.  To solve this problem, we need to assume that $f_T(0, t)$ exists.
	\item Vasicek's model.  Take $d = 1$ and $\sigma(t, T) = \sigma e^{-b(T-t)}$ where $\sigma, b$ are positive numbers.  Then \[
		b(t, T) = -\sigma^\star(t, T) = - \frac{\sigma}{b} \left(e^{-b(T- t)} - 1\right),
	\]  and other computations are given in the course notes.
\end{enumerate}



% section examples_of_hjm_models (end)

\chapter{Valuation of Options in Gaussian Models} % (fold)
\label{sec:valuation_of_options_in_gaussian_models}
\section{Options on Bonds} % (fold)
\label{sec:options_on_bonds}

% section options_on_bonds (end)
Consider any term structure in which at least some bonds are traded.  If the short term rate process is given then under $\P^\star$, \[
	dB(t, T_i) = B(t, T_i) \left( r_t \, dt + b(t, T_i) \, dW^\star_t \right)
\] where $b(t, T_i)$ is a deterministic function and $0 < T_1 < \dots < T_m$.  If $r$ is not explicitly specified then we should focus on the dynamics of the forward prices, for example \[
	F_B(t, T_i, T_j) = \frac{B(t, T_i)}{B(t, T_j)}, \quad i = 1, \dots, m
\] under the forward measure $\P_{T_j}$.

	How do we value and hedge European bond options with maturity $T$ and the underlying zero coupon bond maturing at $U > T$.  The payoff at $T$ equals \begin{align*}
		C_T &= \left( B(T, U) - K \right)^+ \\
		P_T &= \left( K - B(T, U) \right)^+ 
	\end{align*} so that \[
		C_T - P_T = B(T, U) - K
	\] and thus for $t \in [0, T]$, \[
		C_t - P_t = B(t, U) - KB(t, T). 
	\]  Instead of computing the expectation under $\P^\star$, \[
		C_t = B_t \E_{\P^\star} \left( \frac{C_T}{B_T} \given \sigf_t \right),
	\] we will compute the equivalent probability measure $P_T$ \[
		C_t = B(t, T) \E_{\P_T} \left( C_T \given \sigf_t \right).
	\]  Let $D = \left\{ B(T, U) > K \in \sigf_T \right\}$.  Then \[
		C_T = B(T, U) \mathbf{1}_D - K \mathbf{1}_D = X_1 - X_2
	\] So that \[
		C_t = \pi_t(X_1) - \pi_t(X_2) = I_1 - I_2.
	\]  For $I_2$, we compute \begin{align*}
		I_2 = \pi_t(K \mathbf{1}_D) = K B(t, T) \P_T\left(D \given \sigf_T \right).  
	\end{align*} We observe that \[
		B(T, U) = \frac{B(T, U)}{B(T, T)} = F_B(T, U , T)
	\] where under $\P_T$ the forward price $F_B(t, U, T), [t \in [0, T]]$ satisfies \[
		dF_B(t, U , T) = F_B(t, U, T) \left( b(t, U) - b(t, T) \right) \, dW^T_t
	\] so that $F_t = F_B(t, U, T)$ satisfies \[
		F_T = F_t \exp \left( \zeta(t, T) - \frac{1}{2} v^2( t, T) \right)
	\] where \[
		\zeta(t, T) = \int_0^T \gamma(u, U , T) \, dW^T_u, \quad v^2(t, T) = \int_t^T \left| \gamma(u, U, T) \right|^2 \, du
	\] where $\gamma(u, U, T) = b(u, U) - b(u, T)$.    
	
	We need to compute \begin{align*}
		\P_T(D \given \sigf_t) &= P_T(B(T, U) > K  \given \sigf_t) \\
						&= \P_T ( F_B(T, U, T) > K \given \sigf_t) \\
						&= \P_T \left(F_t e^{\zeta(t, T) - \frac{1}{2} v^2(t, T)} \given \sigf_t \right),
	\end{align*} where $\zeta(t, T)$ is independent of $\sigf_t$ and $\zeta(t, T) \sim N(0, v^2(t, T))$.  Hence \begin{align*}
		\P_T(D \given \sigf_t) &= \P_T \left( F e^{\zeta(t, T) - \frac{1}{2} v^2(t, T)} \given F = F_t \right)  \\
		&= \P_T \left( \frac{\zeta(t, T)}{v(t, T)} > \ln \frac{K}{F} + \frac{1}{2} v^2(t, T) \given F = F_t \right) \\
		&= N( \tilde d_{-}(F_t, t, T))
	\end{align*} where $\tilde d_2(F_t, t, T) = \frac{\ln \frac{F}{K} \pm \frac{1}{2} v^2(t, T)}{v(t, T)}$.  
	
	For $I_1,$ we need to compute the conditional expectation \[
		I_1 = B(t, T) \E_{\P_T} \left( B(T, U) \mathbf{1}_D \given \sigf_t \right)
	\] where \[
		\frac{B(T, U)}{C} = \frac{F_B(T, U, T)}{C} = \frac{d\tilde \P_T}{d\P_T}.
	\] so that \begin{align*}
		\frac{d \tilde \P_T}{d \P_T} \given \sigf_t &= \frac{F_B(t, U, T)}{C} \\
			&= \exp \left( \int_0^t \gamma(u, U, T) \, dW^T_u - \frac{1}{2} \int_0^t \left| \gamma(u, U, T) \right|^2 \, du \right) \\
			&= \tilde \eta_t
	\end{align*} for $t \in [0, T]$.  Note also that \begin{align*}
		\E_{\tilde \P_T} (X \given \sigf_t) = \frac{\E_{\P_T}(X \tilde \eta_t \given \sigf_t)}{\tilde \eta_t} \\
		\frac{F_B(t, U, T)}{c} \E_{\tilde \P_T}(\mathbf{1}_D \given \sigf_t) &= \E_{\P_T} \left( \mathbf{1}_D \frac{B(T, U)}{C} \given \sigf_t \right)
	\end{align*} and \begin{align*}
		\E_{\P_T} \left(B(T, U) \mathbf{1}_D \given \sigf_t \right) = \frac{B(t, U)}{B(t, T)} \tilde P_T (D \given \sigf_t) 
	\end{align*} and thus \[
		I_1 = B(t, U) \tilde P_T(D \given \sigf_t)
	\] 
	and since $dF_t = F_t \gamma(t, U , T) dW^T_t$ and \[
		\tilde W_t^T - \int_0^t \gamma(u, U, T) \, du
	\] is a $\tilde P_T$-Brownian motion, we obtain \[
		dF_t = F_t \left( \left|\gamma(t, U, T) \right|^2 \, dt + \gamma(t, U, T) \, d\tilde W^T_t \right)
	\] under $\tilde P_T$.  Solving this equation, we obtain \[
		F_T = F_t \exp \left( \int_t^T \gamma(u, U, T) \, d\tilde W^T_t + \frac{1}{2} \int_0^T \left| \gamma(t, U, T) \right|^2 \, du \right).
	\] and so \[
		\tilde \P_T(D \given \sigf_t) = N(\tilde d_{+}(F_t, t, T)).
	\]
	
	We conclude that \begin{align*}
		I_1 &= B(t, U) N(\tilde d_+(F_t, t, T)), \\
		I_2 &= K B(t, T) N(d_-(F_t, t, T)).
	\end{align*} so that the price of the call bond option is now known explicitly.  It remains to find out whether the call option can be replicated, for instance, by a trading strategy $\phi = (\phi^1, \phi^2)$ with the wealth process $V(\phi)$, \begin{align*}
		V_t(\phi) 	&= \phi_t^1 B(t, U) + \phi_t^2 B(t, T) \\
		dV_t(\phi) 	&= \phi_t^1 dB(t, U) + \phi^2_t dB(t, T) \\
		V_T(\phi) 	&= C_T = (B(T, U) - K)^+.
	\end{align*}  Equivalently,  \begin{align*}
		\frac{V_t(\phi)}{B(t, T)} &= \phi^1_t F_B(t, U, T) + \phi_t^2 \\
		d \left( \frac{V_t(\phi)}{B(t, T)} \right) &= \phi^1_t dF_B(t, U, T) \\
		\frac{V_T(\phi)}{B(T, T)} &= (F_B(T, U, T) - K)^+.  
	\end{align*}

	Let $F_V(t, T) = \frac{V_t(\phi)}{B(t, T)}$.  Then we need to solve the following problem \begin{align*}
		dF_V(t, T) = \phi_t^1 d F_B(t, U, T) \\
		F_V(T, T) = (F_B(T, U, T) - K)^+
	\end{align*} where \[
		dF_B(t, U, T) = F_B(t, U, T) \gamma(t, U, T) dW^T_t.
	\]  To solve this equation, observe that \[
		\frac{C_t}{B(t, T)} = \frac{B(t, U)}{B(t, T)}\left( N( \tilde d_+ (F_t, t, T)) - K N (\tilde d_-(F_t, t, T)) \right), 		
	\] and \[
		F_C(t, T) = F_t \left( N( \tilde d_+ (F_t, t, T)) - K N (\tilde d_-(F_t, t, T)) \right).
	\]

\begin{lem}
	Let $(Y_t)$ be given by \begin{align*}
		Y_t &= X_t \left( N( \tilde d_+ (X_t, t, T)) - K N (\tilde d_-(X_t, t, T)) \right) \\
		dX_t &= X_t \sigma(t) \, dW_t \\
		\tilde d_{\pm}(x, t, T) &= \frac{ \ln \frac{x}{K} \pm ^2(t, T)}{v(t, T)}. 
	\end{align*}
	
	Then \[
		dY_t = N(d_{-}(X_t, t, T)) \, dX_t.
	\]
\end{lem}

\begin{proof}
	Apply the \ito formula.  Assume here that $\sigma$ is deterministic. 
\end{proof}

 If we apply the lemma to $F_c(t, T)$, we obtain \begin{align*}
	dF_c(t, T) &= N(\tilde d_+(F_t, t, T)) \, dF_t \\
				&= \phi^1_t \, dF_t.
\end{align*} so that \[
	\phi^1_t = N(\tilde d_1(F_t, t, T))
\] and \[
	\phi^2_t = \frac{C_t - \phi^1_t B(t, U)}{B(t, T)}
\] Then \begin{align*}
	V_t(\phi) 	&= C_t = \phi^1_tB(t, U) + \phi^2_t B(t, T). \\
	dV_t(\phi) 	&= dC_t = \phi^1_t dB(t, U) + \phi^2-t dB(t, T).
\end{align*}

In the future, we will deal with more general options of the form \[
	C_T = (Z^1_T - K Z^2_T)^+
\]  where $Z^i$ is some portfolio of bonds.  Then the choice of a natural hedging strategy depends on the choice of traded assets.  

\begin{lem}
	The price $C_t$ of a call option equals \[
		C_t = B(t, U) \P_U(D \given \sigf_t) - K B(t, T) \P_T(D \given \sigf_t)
	\]
\end{lem}

\begin{proof}
	\begin{align*}
		C_T = B(T, U) \mathbf{1}_D - K \mathbf{1}_D = X_1 - X_2 \\
		\pi_t(X_2) = B(t, T) \E_{\P_T} K \mathbf{1}_D \given \sigf_t = K B(t, T) \P_T (D \given \sigf_t)
	\end{align*} and $X_1 = B(T, U) \mathbf{1}_D$ is equivalent to $Y_1 = \mathbf{1}_D$ at time $U$, so that \begin{align*}
		\pi_t(X_1) = \pi_t(Y_1) = B(t, U) \P_{U}(D \given \sigf_t)
	\end{align*} for $t \in [0, T]$.
\end{proof}

\section{Options on Coupon Bonds} % (fold)
\label{sec:options_on_coupon_bonds}
Let $T_1 < T_2 < \dots < T_n \leq T^\star$ be coupon dates and $c_1 \dots, c_2$ he corresponding deterministic coupons.  Then the price $Z_t = B_c(t, T)$ of the coupon bond equals \[
	Z_t = \sum_{j=1}^n c_j B(t, T_j).
\]  We consider the call option with maturity $T < T_1$ and the payoff \[
	C_T = (Z_T - K)^+ = \sum_{j=1}^m c_j B(t, T_j) \mathbf{1}_D - K \mathbf{1}_D
\]   where \[
	D = \{ Z_T > K \}.
\]

One possible way of pricing this is to represent $C_t$ as follows: \begin{align*}
	C_t = \sum_{j=1}^n c_j B(t, T_j) \P_{T_j}(D \given \sigf_t) - K B(t, T)P_T(D \given \sigf_t).
\end{align*}

\begin{rem}[On the proof of Proposition 4.3]
	We know that if we set $D = \{ Z_T > K \}$, then \[
		C_t = \sum_{j=1}^m c_j B(t, T_j) \P_{T_j}(D \given \sigf_t) - K B(t, T) \P_T(D \given \sigf_t).
	\]
	For simplicity, we may set $t = 0$ - then we need to compute $\P_{T_j}(D)$ and $\P_T(D)$.  Recall that $T = T_0 < T_1 \dots < T_m$.  Then \[
		D = \left\{ \sum_{j=1}^m c_j \underbrace{F_B(T, T, T_j)}_{F^j_B(T)} > K \right\}
	\]       where $dF^j_B(t) = F^j_B(t)(b(t, T_j) - b(t, T)) dW^T_t$, and hence \[
		\P_T(D) = \P_T \left( \sum_{j=1}^m c_j F^j_B(0) e^{\int_0^T \gamma(t, T_j, T) \, dW^T_t - \frac{1}{2} \int_0^T \left| \gamma(t, T_j, T) \right|^2 \, dt} > K \right)
 	\]
If we denote $\zeta_j = \int_0^T \gamma(t, T_j, T) \, dW^T_t$, then the vector $\zeta = (\zeta_1, \dots, \zeta_m)$ has a normal distribution under $\P_T$, with mean $(0, 0, \dots, 0)$ and covariance $(\nu_{kl})$ where \[
	\nu_{kl} = \int_0^T \gamma(t, T_k, T) \cdot \gamma(t, T_l, T) \, dt.
\]

To compute $\P_{T_j}(D)$, we need to know the distribution of $\zeta$ under $\P_{T_j}$.  Since $W^{T_j}_t = W^T_t - \int_0^t \gamma(u, T_j, T) \, du$ it is clear that under $\P_{T_j}$, the forward price $F^l_B(t) = F_B(t, T_l, T)$ \[
	dF^l_B(t) = F^l_B(t) \gamma(t, T_l, T) \, dW^T_t + F_B(t) \gamma(t, T_l, T) \gamma(t, T_j, T) \, dt
\] so that the joint distribution of $\zeta_1, \dots, \zeta_m$ under each forward measure $\P_{T_j}$ can also be computed.  The joint distribution is Gaussian with the same covariance matrix but with means $v_{lj}$

\end{rem}

\section{Pricing of General Contingent Claims} % (fold)
\label{sec:pricing_of_general_contingent_claims}
Let $\zeta_i(t, T) = \int_t^T \gamma_i(u, T) \, dW^T_u$.  Then under $\P_T$ the random variables $\zeta_i(t, T), \dots, \zeta_n(t, T)$ are normally distributed with mean $(0, \dots, 0)$ and covariance matrix ($\gamma_{ij}$) given by \[
	\gamma_{ij} = \int_t^T \gamma_i(u, T) \gamma_j(u, T) \, du.
\]

\begin{prop}
	Let $X = g(Z^1_T, \dots, Z^n_T)$ at time $T$.  Then the price of $X$ at time $t \in [0, T)$ is given by \[
		\pi_t(X) = B(t, T) \int_{\R^k} g\left( \frac{Z^1_t}{B(t, T)} \frac{n_k(x+ \theta_1)}{n_k(x)}, \dots, \frac{Z^n_t}{B(t, T)} \frac{n_k(x+\theta_n)}{n_k(x)}\right) n_k(x) \, dx
	\] where $n_k$ is the standard $n$-dimensional Gaussian density on $\R^k$ and $(\theta_i)$ are elements of $\R^k$ such that \[
		\theta_i \theta_j = \gamma_{ij}
	\] for all $i, j$.   This follows from the Cholesky decomposition of the covariance matrix $(\gamma_{ij})$
\end{prop}

\begin{proof}
	\begin{align*}
		\pi_t(X) = B(t, T) \E_{\P_T} \left( g(F_{Z^1}(T, T)), \dots, F_{Z^n}(T, T) \given \sigf_t \right) \\
		F_{Z^i}(T, T) = F_{Z^i}(t, T) e^{\zeta_i(t, T) - \frac{1}{2} \gamma_{ii}} \\
		\pi_t(X) = B(t, T) \E_\Q \left( g\left( F_{Z_t^i} e^{\theta_i \eta - \frac{1}{2} \gamma_{ii}} \right) \given \sigf_t \right) \\
		&= B(t, T) \int_{\R^k} g\left( \frac{Z^i_t}{B(t, T)} e^{\theta_i \cdot x - \frac{1}{2} \underbrace{\gamma_{ii}}_{|\theta_i|^2}} \right) n_k(x) \, dx.
	\end{align*}  Since $\frac{n_k(x + \theta_i)}{n_k(x)} = e^{\theta_i \cdot x - \frac{1}{2}|\theta_i|^2}$, we obtain our result.
\end{proof}
% section pricing_of_general_contingent_claims (end)
% section options_on_coupon_bonds (end)
% section valuation_of_options_in_gaussian_models (end)
% chapter the_hjm_approach_to_modelling_bond_prices (end)

\chapter{Modelling of Forward LIBORs} % (fold)
\label{cha:modelling_of_forward_libors}
\section{Introduction to LIBOR} % (fold)
\label{sec:introduction}
Let $\delta$ equal 3 months.  If $L(0) = 10\%$ then if we borrow $N$ at time $0$, we will pay back after three months the amount $N(1 + \delta L(0))$ where the unit is one year so that $\delta = \frac{1}{4}$.

\begin{enumerate}[(i)]
	\item Spot LIBOR is (or was) the most commonly used rate for interbank funding and as an underlying for interest rate derivatives such as caps and floors.  
	\item By convention, the pricing formula for caplets and floorlets was a version of the Black formula which reads \[
		C_T = F_t N(d_+) - K N(d_-)
	\] where $F_t$ is the forward price of the underlying asset. 
\end{enumerate}


Let us consider a caplet with maturity $T$ and settlement date $T + \delta$.  Here, a caplet is a call option on LIBOR, in the sense that it pays the amount $C_P = (L(T) - K)^+ \delta N$ at time $T + \delta$ where $T$ is the maturity date, $N$ is the nominal value, and $x$ the strike level.  

\begin{defn}[Cap]
	 A cap is a portfolio of caplets over non-overlapping periods \[
	 	0 < T_0 < T_1 < \dots < T_n 
	 \] so we have $n$ caplets, struck at $T_{i}$ for the period $[T_i, T_{i+1}]$ and paying $(L(T_i) - K)^+ N \delta_{i+1}$ at $T_{i+1}$, where $\delta_{i+1} = T_{i+1} - T_i$.
\end{defn}


By convention, the price of a caplet over $[T, T+\delta]$ equals \[
	\textsc{Cpl}_t = B(t, T+\delta) \left( L(t) N(d_+) - KN(d_-)\right)
\] where \[
	d_\pm = \frac{\ln \frac{L(t)}{K} \pm \frac{1}{2} \sigma^2(T-t)}{\sigma \sqrt{T-t}}.
\]


  
% section introduction (end)

\section{Caps and Floors in the LIBOR Market Model} % (fold)
\label{sec:caps_and_floors_in_the_libor_market_model}

A caplet (floorlet) is a protection against the rise (fall) in the LIBOR rate. The caplet (floorlet) pays off: \begin{align*}
 	\mathbf{Cpl}^j_{T_j} 	N\left( L(T_{j-1}) - \kappa\right)^+ \delta_j \\
	\mathbf{Frl}^j_{T_j} 	N\left(  \kappa - L(T_{j-1}) \right)^+ \delta_j \\
\end{align*} paid at time $T_j$.  

We clearly have the \emph{cap-floor put call parity}, \[
	\mathbf{Cpl}^j_{T_j} - \mathbf{Frl}^j_{T_j} = N_p \left(L(T_{j-1}) - \kappa \right) \delta_j. 
\]  

\begin{exer}
	Using this relationship, find the difference $\mathbf{CPl}_t^j - \mathbf{Frl}^j_t$ for any $t \in [0, T_{j-1}]$.
\end{exer}

Recall that \[
	1 + \delta_j L(T_{j-1}) = \frac{1}{B(T_{j-1}, T_j)}
\]  Hence \begin{align*}
	\mathbf{Cpl}^j_{T_j} = N \left( \frac{1}{B(T_{j-1}, T_j)} - \underbrace{(1 + \delta_j \kappa)}_{\tilde \delta_j} \right)^+ \delta_j
\end{align*}  An equivalent payoff at time $T_{j-1}$ equals \begin{align*}
	\tilde{\mathbf{Cpl}}^j_{T_{j-1}} &= B(T_{j-1}, T_j) \mathbf{Cpl}_{T_j}^j \\
				&= \tilde \delta_j N \left( \frac{1}{\tilde \delta_j} - B(T_{j-1}, T_j) \right)^+.
\end{align*}

% section caps_and_floors_in_the_libor_market_model (end)
% chapter modelling_of_forward_libors (end)


\begin{defn}
	The forward swap rate $\kappa(t, T_0, T_1, \dots, T_n) = \kappa(t, T, n)$ where $T_0 = T$ is the $\sigf_t$-measurable random variable such that $\mathbf{FS}_t \left( \kappa(t, T, n) \right) = 0$.  
\end{defn}

\begin{lem}
	The forward swap rate equals \begin{align*}
		\kappa(t, T, n) = \frac{B(t, T_0) - B(t, T_n)}{\sum_{j=1}^n \delta_j B(t, T_j)}
	\end{align*}
\end{lem}

\chapter{Modelling of Forward Swap Rates} % (fold)
\label{cha:modelling_of_forward_swap_rates}
\begin{enumerate}[(i)]
	\item Definition and payoffs of an $n$-period forward swap.
	\item Valuation formula for a forward swap (6.4)
	\item Definition and formula for forward swap rates (6.5)
	\item Definition and equivalent representations for a swaption (Lemma 6.5)
	\item Postulates of Jamshidian's model of co-terminal forward swap rates
	\item Valuation of a swaption (Proposition 6.3)
	\item Choice of a numeraire portfolio
\end{enumerate}

Consider the family of co-terminal swap rates \begin{align*}
	\kappa(t, T_0; n) &= \frac{B(t, T_0) - B(t, T_n)}{\sum_{k=1}^n \delta_k B(t, T_k)} \\
	\kappa(t, T_1,; n-1) &= \frac{B(t, T_1) - B(t, T_n)}{\sum_{k=2}^n \delta_k B(t, T_k)} \\
	&\downarrow \\
	\kappa(t, T_{n-1}; 1) &= \frac{B(t, T_{n-1}) - B(t, T_n)}{\delta_n B(t, T_n)} = L(t, T_{n-1}) 
\end{align*}  For ease of notation, we let $\kappa(t, T_; n-j) = \tilde \kappa(t, T_j)$.

\section{Payer Swaptions} % (fold)
\label{sec:payer_swaptions}
Let us take $j = 0$ so that the underlying forward swap has $n$ periods.  Let $\mathbf{FS}_t(\kappa)$ denote the value of the forward swap. We know that \[
	\mathbf{FS}_t(\kappa) = B(t, T_0) - \sum_{j=1}^n c_j B(t, T_j)
\] where $c_j = \kappa \delta_j$, $j = 1, \dots, n-1$, $c_n = (1 + \kappa \delta_n)$.  

\begin{lem}
	The price $\mathbf{FS}_t(\kappa)$ can be represented as follows: \begin{align*}
		\mathbf{FS}_t(\kappa) &= \mathbf{FS}_t(\kappa) - \mathbf{FS}_t(\kappa(t, T_0; n)) \\
					&= \sum_{j=1}^n \left( \kappa(t, T_0; n) - \kappa \right) \delta_j B(t, T_j) \\
					&= G_t(n)
	\end{align*}  where \[
		G_t(n) = \sum_{\delta_k B(t, T_k)}, \quad G_t(n-j) = \sum_{k=j+1}^n \delta_k B(t, T_k)
	\]  
\end{lem}

A payer swaption with a fixed rate $\kappa$, maturing date $T = T_0$ and the underlying $n$-period fixed-for-floating forward swap can be identified with the payoff $\left(\mathbf{FS}_T(\kappa) \right)^+$ at time $T$.  A receiver swaption pays $\left(-\mathbf{FS}_T(\kappa) \right)^+$ at time $T$.  Of course, we have a put call parity relationship \[
	\mathbf{PS}_t(\kappa) - \mathbf{RS}_t(\kappa) = \mathbf{FS}_t(\kappa)
\]

The inequality $\mathbf{FS}_t(\kappa) > 0$ holds if and only if $\kappa(T, T; n) > \kappa$ where $\kappa(T, T; n)$ is the spot swap rate at time $T_0$. Hence if $\kappa(T, T; n) \leq \kappa$ the swaption expires worthless, but it is still possible to enter at $T$ a forward swap with fixed rate $\kappa(T, T; n) \leq \kappa$.

If we define \[
	Y_k = \delta_k \left( \kappa(T, T; n) - \kappa \right)^+,
\] we know that \begin{align*}
	\left( \mathbf{FS}_t(\kappa) \right)^+ &= \sum_{k=1}^n \delta_k B(T, T_k) \left(\kappa(T, T; n) - \kappa \right)^+ \\
	&= \sum_{k=1}^n B(T, T_k) Y_k
\end{align*} which is equivalent to a sequence of payoffs $Y_1, \dots, Y_n$ at times $T_1, \dots, T_n$.  Also for $j = 0, 1, \dots, n-1$, \begin{align*}
	\left( \mathbf{FS}_{T_0}^0 (\kappa) \right)^+ = G_{T_0}(n) \left( \kappa(T_0, T_0; n) - \kappa \right)^+ \\
	\left( \mathbf{FS}_{T_j}^j (\kappa) \right)^+ = G_{T_j} (n-j) \left( \kappa(T_j, T_j) ;n) - \kappa \right)^+
\end{align*}

We now seek to construct a model for the joint dynamics of a co-terminal family of forward swap rates \[
	\kappa(t, T_j; n-j) = \tilde \kappa(t, T_j), t \in [0, T_j] 
\] such that the volatility $\nu(t, T_j)$ is given in advance by a deterministic function and the model is driven by a $d$-dimensional Brownian motion.\footnote{Any process that we can apply Girsanov's theorem to will be sufficient.}

We expect that each process $\tilde \kappa(t, T_j)$ will be a martingale under some probability measure $\tilde P_{T_{j+1}}$ so that \[
	d\tilde \kappa(t, T_j) = \tilde \kappa(t, T_j) \nu(t, T_j) \, d\tilde W^{T_{j+1}}_t
\] where $\tilde W_t^{T_{j+1}}$ is a Brownian motion under $\tilde P_{T_{j+1}}$ and the Radon-Nikodym densities for $j=0, \dots, n-1$ should be given by \[
	\frac{d \tilde P_{T_{j+1}}}{d \tilde \P_{T_n}} = \mathbf{?}
\] which should be expressed in terms of $\tilde W^{T_n}, \tilde \kappa(t, T_k), \nu(t, T_k)$ for $k = n-j+1, \dots, n$. 

\section{Valuation of Swaptions in Jamshidian's Model} % (fold)
\label{sec:valuation_of_swaptions_in_jamshidian_s_model}
Let us assume that the model is well defined.  We will value the $j$-th swaption for $j = 0, \dots, n-1$.  Suppose that it is attainable, so that the price can be computed using the martingale method, meaning here that \[
	\pi_t(X) = G_t(n-j) \E_{\tilde \P_{T_{j+1}}} \left( \frac{X}{G_{T_j}(n-j)} \given \sigf_t \right)
\] where $X$ is any attainable claim in Jamshidian's model with maturity $T$.  Observe that only a finite family of forward swaps are traded in this model.  In our case, $X = G_{T_j}(n-j) \left( \tilde \kappa(T_j, T_j) - \kappa \right)^+$, and thus \[
	\mathbf{PS}_t^i(\kappa) = G_t(n-j) \E_{\tilde \P_{T_{j+1}}} \left( \left( \tilde \kappa(T_j, T_j) - \kappa \right)^+ \given \sigf_t \right).
\]  Since $\eta(t, T_j): [0, T_j] \rightarrow \R^d$ is deterministic, we can evaluated this expression using the Black formula, and obtain \[
	\tilde \kappa(t, T_j) \Phi \left( \tilde d_+^j(\tilde \kappa(t, T_j), t, T_j) \right) - \kappa \Phi \left(\tilde d_-^j (\tilde \kappa(t, T_j), t, T_j) \right)
\]

 where \begin{align*}
	\tilde d_{\pm}(x, t, T_j) &= \frac{\ln \frac{x}{\kappa} \pm \frac{1}{2} v^2_j(t, T_j)}{v_j(t, T_j)} \\
	v_j(t, T_j) &= \int_t^{T_j} \left| v(u, T_j) \right|^2 \, du.
\end{align*}


For replication of a swaption, we formally define the relative price \[
	\mathbf{F}_{S_j, G}(t, T_j) = \frac{\mathbf{PS}_t^j}{G_t(n-j)} = \tilde \kappa(t, T_j) \Phi \left( \tilde d_+^j(t) \right) - \kappa \Phi(\tilde d^j_-(t)).
\]  In this case, \[
	dF_{S_j, G}(t, T_j) = \Phi(\tilde d_+^j(t)) \, d \tilde \kappa(t, T_j).
\]

It is possible to then hedge this option using forward swaps in discrete time.  

Let $\psi^j$ be any trading strategy in the $j$-th forward swap.  At time $0$ the value of our strategy is zero.  Then the trading strategy: \begin{align*}
	t = 0 & \quad \text{$\psi^j_0$ positions in market forward swap with rate $\tilde \kappa(0, T_j)$} \\
	t = t_1  & \quad \text{$\phi^j_{t_1}$ positions in market forward swap with rate $\tilde \kappa(t_1, T_j)$} \\
		&\downarrow
	t = t_n = T_j
\end{align*}  Then gains and losses can be conveniently expressed in units of $G_t(n-j)$.  For instance, the value of our $\psi^j_0$ positions at time $t_1$ equals \begin{align*}
	\mathbf{PL}_{t_1} = G_{t_1}(n-j) \psi^j_0 \left( \tilde \kappa(t_1, T_j) - \tilde \kappa(0, T_j) \right) \\
	\tilde{\mathbf{PL}}_{t_1} = \underbrace{\psi^j_0 \left( \tilde \kappa(t_1, T_j) - \tilde \kappa(0, T_j) \right)}_{\text{paid in installments at times $T_{j+1}, \dots, T_n$.}}
\end{align*}  After $n$ steps, \begin{align*}
	\tilde{\mathbf{PL}}_{T_j} = \sum_{k=0}^{n-1} \psi_{t_k}^j \left( \kappa(t_{k+1}, T_j) - \tilde \kappa(t_k, T_j) \right) \\
	&\rightarrow_{\substack{n \rightarrow \infty \\ t_k = \frac{k}{n} T_j}} \int_0^{T_j} \psi^j_u \, d\kappa(u, T_j) 
\end{align*}

The premium $\mathbf{PS}_0^j$ is totlaly invested in the level porftolio $G(n-j)$ so that the totla value of the profit and loss at time $T_j$ equals \[
	\frac{\mathbf{PS}^j_0}{G_0(n-j)} + \int_0^{T_j} \psi^j_u \, d \kappa(u, T_j)
\]  Taking derivatives, we can show that by setting $\psi^j_t = \Phi \left( \tilde d_+^j(t) \right)$ we obtain the replicating strategy for the $j$-th swaption.
% section valuation_of_swaptions_in_jamshidian_s_model (end)
% section payer_swaptions (end)
% chapter modelling_of_forward_swap_rates (end)
\end{document}























